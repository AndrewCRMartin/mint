\documentclass[12pt]{article}
\usepackage{a4}
\newcommand{\appsection}[1]{\par\vspace{1ex}\noindent{\bfseries #1}\par\noindent}

\title{MINT V3.2}
\author{Dr. Andrew C.R. Martin, UCL}
\date{Copyright (c) 1995--2006}

\begin{document}
\maketitle

MINT is a graphical interface to the basic modelling functions of
Andrej \v{S}ali's Modeller program. This interface will simply allow you
to perform homology modelling; none of the other capabilities of
Modeller is supported. 

MINT may be used to perform `full homology' modelling where you simply
give Modeller the PDB files to use as templates and Modeller
calculates a sequence alignment. Alternatively, you may provide an
alignment and request Modeller to work from this. The latter protocol
is recommended since an accurate alignment is crucial to obtaining a
good model.

If MINT and/or Modeller have not been globally installed at your
site, please read Appendix~\ref{app:install}.


\section{Full Homology Modelling}
In full homology modelling, you need only supply the sequence of the
protein you wish to model and the PDB code of the template structures
you wish to use.
\vspace{1em}

\noindent Proceed as follows:

1. Create a file containing the sequence of the protein to be modelled
in standard PIR format. This format is described in
Appendix~\ref{app:pir}

2. Start MINT by typing {\tt mint} at the Unix prompt.

3. In the text box labelled {\bfseries PIR Sequence File:}, enter the
filename of the PIR sequence file you have created which contains the
sequence of the protein to be modelled. If this box is grey and
inactive (you can't type in it even after clicking in it with the
mouse), click the {\bfseries Specify Alignment} button and then click
the {\bfseries No Alignment} button in the window which appears.

4. In the text box labelled {\bfseries Template PDB codes:}, enter the
PDB codes of the structures you are using as templates for the
homology model building. You should enter just the 4 character PDB
code (e.g.\ 1crn), no preceeding characters (such as {\tt pdb} or {\tt
p}) are required. Case is important; you should use the same case as
used in your PDB directory where the PDB files are stored since these
codes are used directly to access PDB files. This will generally be
lower case.

Optionally, you may specify a 5th character in the PDB code. This is a
chain label allowing you to model a single chain of a multi-chain protein.

If you are using more than one template structure, the PDB codes
should be separated by spaces.

MINT will {\bfseries first look in the current directory} for PDB
files. {\bfseries Note that any PDB files in your current directory
must have the same prefix and extension as PDB files in the main PDB
directory.} (See Appendix~\ref{app:system}.)

5. Select the number of models required and the refinement method you
wish to use.

6. Click the {\bfseries Run} button to start the Modeller program and to exit
MINT. If you wish to exit without running Modeller, click the {\bfseries Quit}
button. 


\section{Alignment-Specified Modelling}
In alignment-specified modelling, you must provide a sequence
alignment containing the sequence of the protein to be modelled with
the template structures you are going to use. This is the recommended
protocol since a good alignment is critical in obtaining an accurate
model. 
\vspace{1em}

\noindent Proceed as follows:

1. Create a file containing the sequence of the protein to be modelled
aligned with the sequences of the template structures in Modeller's
PIR-like alignment format. This format is described in
Appendix~\ref{app:align}.

2. Start MINT by typing {\tt mint} at the Unix prompt.

3. Click the {\bfseries Specify Alignment} button. A new window will
appear.

4. In the text box labelled {\bfseries Alignment file:}, enter the
name of the alignment file you have created. Ensure that the PIR radio
button is active rather than the Quanta button.

5. Enter the identifier of the sequence to be modelled in the text box
labelled {\bfseries Target code:}.

6. Click on the button labelled {\bfseries Exit}. The new window will
disappear.

7. In the text box labelled {\bfseries Template PDB codes:}, enter the
identifiers used in the alignent file (Appendix~\ref{app:align}) for
the structures you are using as templates for the homology model
building.

If you are using more than one template structure, the codes
should be separated by spaces.

Modeller will look for the PDB files in the current directory and the
main PDB directory (see Appendix~\ref{app:system}) using filenames
which are specified in the alignment file (see
Appendix~\ref{app:align}). The codes entered here are not used
directly to find the PDB files.

8. Select the number of models required and the refinement method you
wish to use.

9. Click the {\bfseries Run} button to start the Modeller program and to exit
MINT. If you wish to exit without running Modeller, click the {\bfseries Quit}
button. 




\section{Advanced Options}
Clicking the {\bfseries Advanced options} button brings up a window in
which three options may be set. These control the reading and writing
of hetero (non-protein) atoms, waters and hydrogens. By default, all
three of these are ignored.

In the advanced options window, clicking one of the three checkboxes
will toggle the appropriate option on or off.

Click the {\bfseries Exit} button at the bottom of the window to close
it.



\section{Other Information}

Alignment files may also be specified in Quanta format. These may be
generated from within the Quanta software by following the Quanta
protocols for Homology Modelling (note that you do not need the MSI
version of Modeller to do this). If you generate a Quanta alignment
file, you simply specify this in the {\bfseries Alignment File:} text
box and click the {\bfseries Quanta} radio button.

If you use a Quanta format alignment file, the codes used in the file
and in the {\bfseries Target code:} and {\bfseries Template PDB
codes:} text boxes must be valid PDB file stems (e.g.\ p1crn if your
PDB files are stored as {\tt /pdb/p1crn.pdb}).






\newpage
\noindent{\huge\bfseries Appendices}

\appendix
\section{Installation}
\label{app:install}
Full installation instructions for Modeller are provided with the
program and installation instructions for MINT are provided in
the INSTALL file in the MINT directory.

If the software has been installed by another user, but not made
globally available, you need to take the following steps:\\
1. Create the environment variables MODINSTALL and MINTDIR to point to
where Modeller and MINT have been installed,\\
2. Create the KEY environment variable used by modeller,
3. Source the Modeller setup script,\\
4. Create an alias to run the MINT program.

Assuming that Modeller has been installed in a directory which is called 
\verb1/home/bsm/martin/sg/bin/modeller31 
and MINT in \verb1/home/bsm/martin/modeller1,
the following script will set everything as required:

\begin{verbatim}
# Customise these 2 for your site...
# ----------------------------------
# Root directory for installed MODELLER:
setenv MODINSTALL /home/bsm/martin/sg/bin/modeller3
# Directory in which MINT is installed
setenv MINTDIR      '/home/bsm/martin/modeller'

if (-e $MODINSTALL) then
   # Root directory for the Protein DataBank (not essential, can be omitted):
   setenv PDB /pdb

   # MODELLER key. Change to the key provided by Andrej Sali:
   setenv KEY <modeller-key-here>

   # Set MODELLER environment variables and update the command path:
   if (-e $MODINSTALL/bin/setmodeller) source $MODINSTALL/bin/setmodeller

   # Read modeller HTML docs using Mosaic
   alias moddoc 'mosaic $MODINSTALL/doc/manual/manual.html'

   # Alias for mint
   alias  mint         $MINTDIR/mint.tcl
endif
\end{verbatim}


\noindent{\bfseries If you are at BSM Unit, UCL} simply add the
following to your .cshrc file:

\begin{verbatim}
source /home/bsm/martin/modeller/setmodeller
\end{verbatim}







\section{PIR File Format}
\label{app:pir}
The PIR sequence file format consists of two header lines followed by
the amino acid sequence using 1-letter code ending with an asterisk
(*).

The first header line is of the form:
\begin{verbatim}
>P1;xxxxxx
\end{verbatim}
where {\tt xxxxxx} is an identifier for the sequence. Any sequence of up to
6 characters may be supplied.

The second header line is a title describing the sequence. Optionally
this may consist of two fields separated by a dash (-). If so, the
second field describes the source of the sequence (e.g.\ `human').

The sequence follows using the standard 1-letter code. Spaces and line
breaks are ignored and an asterisk (*) marks the end of the
sequence. Note that chain breaks are indicated with a slash (/) not an
asterisk as used in the standard PIR format.









\section{Alignment File Format}
\label{app:align}
The alignment file format is an extension of the PIR format described
in Appendix~\ref{app:pir}.

The alignment is simply created by using dash (-) characters to
indicate deletions in the sequences. 

The first header line is a standard PIR header line of the form:
\begin{verbatim}
>P1;xxxxxx
\end{verbatim}
Note that the {\tt xxxxxx} code (up to 6 characters) is used as an
identifier by the {\bfseries Target code:} and {\bfseries Template PDB
codes:} text boxes in MINT.

The second (comment) header line is modified and contains 10 fields
separated by colons (:). These fields have the following meanings:

1. The type of structure associated with the sequence. This is
specified as follows:
\begin{center}
\begin{tabular}{ll}
   sequence    & No structure available,        \\
   structureX  & An X-ray crystal structure,    \\
   structureN  & An NMR structure,              \\
   structureM  & A model structure.             \\
\end{tabular}
\end{center}

2. The filestem of the PDB file containing the associated
structure. Any characters prepended onto the PDB code {\bfseries are included},
but the extension and directory are not. Thus, if we are using crambin
as a structure (PDB code 1crn) and we store the PDB files as
{\tt /pdb/pXXXX.pdb}, this would be specified as {\tt p1crn}. This field is
blank if there is no associated structure (i.e.\ this is the sequence
to be modelled).

3. The residue number (in the PDB file) of the first residue of the
sequence. Normally this will be the first residue number in the PDB
file. This field is blank if there is no associated structure
(i.e.\ this is the sequence to be modelled).

4. The chain name (in the PDB file) of the first residue in the
sequence (or blank).

5. The residue number (in the PDB file) of the last residue of the
sequence. Normally this will be the last residue number in the PDB
file. This field is blank if there is no associated structure
(i.e.\ this is the sequence to be modelled).

6. The chain name (in the PDB file) of the last residue in the
sequence (or blank).

7. The name of the sequence. This is a text description of the protein
and is normally the first COMPND record from a PDB file. This field
may be left blank.

8. The source of the protein (e.g.\ `HUMAN', `MOUSE'). this field may
be left blank.

9. The resolution of the crystal structure. This field is set to 0.00
if there is no associated structure (i.e.\ this is the sequence to be
modelled) or this is an NMR structure.

10. The R-factor of the crystal structure. This field is set to 0.00
if there is no associated structure (i.e.\ this is the sequence to be
modelled) or this is an NMR structure.









\section{System Options}
\label{app:system}
System options for MINT should be set at installation time. However, these
may be overridden once MINT has been started by clicking on the
{\bfseries System} button.
\vspace{1em}

\noindent These options have the following meanings:

\appsection{PDB Directory}
\noindent This is the location of the main store of PDB files. MINT
and Modeller will always look in the current directory for PDB files
first and will then look in the directory specified here. Note that a
trailing slash (/) must be specified. 

Note that the PDB filename will be created from the 4-letter PDB code
you specify plus the prefix and extension (as described
below). \emph{Thus PDB files in your current directory {\bfseries
must} use the same naming convention as those in the main PDB
directory} when doing full homology modelling.

\appsection{PDB Extension}
\noindent This is the filename extension used for your PDB files. Note
that this must include a dot (.) if your PDB files have a dot before
the extension.

\appsection{PDB Prefix}
\noindent This is any character(s) prepended onto the PDB code to
create the filename. If your site has no prepended characters, place a
- in this field.



\subsection{Examples}
\noindent The follwing examples show how the PDB file with PDB code
2hfl are stored and how these values should be set:

\begin{center}
\begin{tabular}{ll}
Name format        &    /pdb/p2hfl.pdb          \\
PDB directory      &    /pdb/                   \\
PDB extension      &    .pdb                    \\
PDB prefix         &    p                       \\
                   &                            \\
Name format        &    /data/pdb/pdb2hfl.ent   \\
PDB directory      &    /data/pdb/              \\
PDB extension      &    .ent                    \\
PDB prefix         &    pdb                     \\
\end{tabular}
\end{center}

\subsection{The current directory}
\noindent When looking in the current directory for PDB files, the
filename is built from the prefix, PDB code and extension. Therefore,
{\bfseries PDB files in your current directory must have the same prefix and
extension as files in the main PDB directory}.




\section{Frequently Asked Questions}

{\bfseries 1. Q:} When doing full homology modelling using a PDB file
in the current directory, why does MODELLER crash with a message of
the form:
\begin{verbatim}
TOP______>    84  262 READ_ALIGNMENT FILE = SEGFILE, ALIGN_CODES = KNOWNS
rdseqpd_E> too many residues
recover__>  MODELLER_STATUS  >=  STOP_STATUS:   1  1
\end{verbatim}

\noindent{\bfseries A:} MODELLER is actually telling you that it cannot find
the PDB file you specified for the template! The most probably reason
for this is that the filenaming convention you have used for the file
in the current directory doesn't match that in the PDB.
\vspace{1em}

\end{document}
