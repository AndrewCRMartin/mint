\section{PIR File Format}
\label{app:pir}
The PIR sequence file format consists of two header lines followed by
the amino acid sequence using 1-letter code ending with an asterisk
(*).

The first header line is of the form:
\begin{verbatim}
>P1;xxxxxx
\end{verbatim}
where {\tt xxxxxx} is an identifier for the sequence. Any sequence of up to
6 characters may be supplied.

The second header line is a title describing the sequence. Optionally
this may consist of two fields separated by a dash (-). If so, the
second field describes the source of the sequence (e.g.\ `human').

The sequence follows using the standard 1-letter code. Spaces and line
breaks are ignored and an asterisk (*) marks the end of the
sequence. Note that chain breaks are indicated with a slash (/) not an
asterisk as used in the standard PIR format.









\section{Alignment File Format}
\label{app:align}
The alignment file format is an extension of the PIR format described
in Appendix~\ref{app:pir}.

The alignment is simply created by using dash (-) characters to
indicate deletions in the sequences. 

The first header line is a standard PIR header line of the form:
\begin{verbatim}
>P1;xxxxxx
\end{verbatim}
Note that the {\tt xxxxxx} code (up to 6 characters) is used as an
identifier by the {\bfseries Target code:} and {\bfseries Template PDB
codes:} text boxes in MINT.

The second (comment) header line is modified and contains 10 fields
separated by colons (:). These fields have the following meanings:

1. The type of structure associated with the sequence. This is
specified as follows:
\begin{center}
\begin{tabular}{ll}
   sequence    & No structure available,        \\
   structureX  & An X-ray crystal structure,    \\
   structureN  & An NMR structure,              \\
   structureM  & A model structure.             \\
\end{tabular}
\end{center}

2. The filestem of the PDB file containing the associated
structure. Any characters prepended onto the PDB code {\bfseries are included},
but the extension and directory are not. Thus, if we are using crambin
as a structure (PDB code 1crn) and we store the PDB files as
{\tt /pdb/pXXXX.pdb}, this would be specified as {\tt p1crn}. This field is
blank if there is no associated structure (i.e.\ this is the sequence
to be modelled).

3. The residue number (in the PDB file) of the first residue of the
sequence. Normally this will be the first residue number in the PDB
file. This field is blank if there is no associated structure
(i.e.\ this is the sequence to be modelled).

4. The chain name (in the PDB file) of the first residue in the
sequence (or blank).

5. The residue number (in the PDB file) of the last residue of the
sequence. Normally this will be the last residue number in the PDB
file. This field is blank if there is no associated structure
(i.e.\ this is the sequence to be modelled).

6. The chain name (in the PDB file) of the last residue in the
sequence (or blank).

7. The name of the sequence. This is a text description of the protein
and is normally the first COMPND record from a PDB file. This field
may be left blank.

8. The source of the protein (e.g.\ `HUMAN', `MOUSE'). this field may
be left blank.

9. The resolution of the crystal structure. This field is set to 0.00
if there is no associated structure (i.e.\ this is the sequence to be
modelled) or this is an NMR structure.

10. The R-factor of the crystal structure. This field is set to 0.00
if there is no associated structure (i.e.\ this is the sequence to be
modelled) or this is an NMR structure.

